\documentclass[margin,line,pifont,palatino,courier]{res}

\usepackage{pifont}
\usepackage[latin1] { inputenc}

%\topmargin .5in
%\oddsidemargin -.5in
%\evensidemargin -.5in
%\textwidth=6.0in
 \textheight=9.0in
%\itemsep=0in
%\parsep=0in
\usepackage{fancyhdr}
%\topmargin=0in
%\textheight=8.5in
\pagestyle{fancy}
\renewcommand{\headrulewidth}{0pt}
\fancyhf{}
%\cfoot{\thepage}
%\lfoot{\textit{\footnotesize Research Statement}}
\rfoot{{\footnotesize Curriculum Vitae, Jared Nishikawa, \thepage}}


\newenvironment{list1}{
  \begin{list}{\ding{113}}{%
      \setlength{\itemsep}{0in}
      \setlength{\parsep}{0in} \setlength{\parskip}{0in}
      \setlength{\topsep}{0in} \setlength{\partopsep}{0in}
      \setlength{\leftmargin}{0.17in}}}{\end{list}}
\newenvironment{list2}{
  \begin{list}{$\bullet$}{%
      \setlength{\itemsep}{0in}
      \setlength{\parsep}{0in} \setlength{\parskip}{0in}
      \setlength{\topsep}{0in} \setlength{\partopsep}{0in}
      \setlength{\leftmargin}{0.2in}}}{\end{list}}

\begin{document}

\name{Jared Nishikawa \vspace*{.1in}}

\begin{resume}

\section{\sc Contact Information}

\vspace{.05in}
\begin{tabular}{@{}p{2.75in}p{2in}}
    \verb+jared.nishikawa@gmail.com+ &\\
    (503)-583-3273 &\\
    
%University of Colorado, Boulder& (503)583-3273 \\
%Department of Mathematics                        & \verb+jared.nishikawa@gmail.com+\\
%Boulder, CO 80309                  & \verb+http://euclid.colorado.edu/~jani1751/+\\
%251 Mercer Street               & \\
%New York, New York 10012 USA               & \\
\end{tabular}

\section{\sc Research Interests}
Graph theory, language parsing, abstract syntax trees, binary analysis, reverse engineering, data classification and clustering, number theory, elliptic curves, cryptography.

\section{\sc Education}

{\bf University of Colorado, Boulder}\\
\vspace*{-.1in}
\begin{list1}
\item[] Ph.D.~in Mathematics, May 2016

\begin{list2}
\vspace*{.05in}
\item Dissertation: \textit{Applications of Cryptographic Hash Functions}
\item Advisor: John Black
\end{list2}
%\item[] M.S.~in Mathematics, May 1996
\end{list1}

{\bf Willamette University}\\
\vspace*{-.1in}
\begin{list1}
\item[] B.A.~in Mathematics, May 2010

\begin{list2}
\vspace*{.05in}
\item Senior thesis: \textit{Hitting Times of the Ising Model}
\end{list2}

\end{list1}

    \section{\sc Research Projects}

    Binary Emulation
    \begin{itemize}
        \item Emulate execution of a PE file
        \item Gather more data about execution compared to static analysis at a fraction of the overhead of dynamic analysis
        \item Use extra data as more input in training vectors for a machine learning model
    \end{itemize}

    Powershell Deobfuscation
    \begin{itemize}
        \item Powershell language analysis (scanner, parser, reducer)
        \item Reduce obfuscated powershell to simple forms
    \end{itemize}

    Malware Classification
    \begin{itemize}
        \item Samples of malware from the same family have patterns, at the opcode level
        \item Break binary executables into control-flow graphs of opcodes
        \item Perform n-gram analysis on graphs to use as a simple vector for clustering algorithms
    \end{itemize}

    Automated Unpacking
    \begin{itemize}
        \item Many malware files are packed (often with UPX) for both size and obfuscation reasons
        \item Many packed executables have the UPX headers stripped: this means the file can still run, but analysis is basically impossible because the upx command will not unpack it
        \item Goal: reverse engineer the UPX packing algorithm to automatically detect the relevant header information from a stripped, packed file, and then auto-unpack the file
        \item This works in tandem with the malware classification project (above) to more precisely classify malware
    \end{itemize}

    \section{\sc Engineering Projects}
    
    Vulnerability Management
    \begin{itemize}
        \item Gather information from asset management sources and normalize data
        \item Create assets and scans in appropriate cloud security data center
    \end{itemize}

    Offensive Security
    \begin{itemize}
        \item Automated scans
        \item Automated patching of critical vulnerabilities
        \item Automated password rotation for internal systems found with common passwords
    \end{itemize}

    Managed Detection
    \begin{itemize}
        \item Tooling for team-wide use
    \end{itemize}

    Threat Analysis
    \begin{itemize}
        \item Tooling for team-wide use
        \item Server management to function as middleware API for common tools
    \end{itemize}


%\section{\sc Teaching Experience}
%
%\textbf{ SecureSet Academy, Denver, CO}
%
%April 2016 - \textit{Present.}  Lead instructor.
%
%\begin{itemize}
%            \setlength\itemsep{0.00em}
%    \item Taught three-week intensive modules in:
%        \begin{itemize}
%            \setlength\itemsep{0.00em}
%            \item[] Applied Cryptography
%            \item[] Introductory Programming in Python
%            \item[] Introductory Network Security
%        \end{itemize}
%    \item Other responsibilities include:
%        \begin{itemize}
%            \setlength\itemsep{0.00em}
%            \item[] Curriculum development and office hours
%            \item[] Coordinating instructors and teaching assignments
%            \item[] Interviewing prospective students and potential instructors
%        \end{itemize}
%\end{itemize}
%
%\textbf{ University of Colorado, Boulder, CO}
%
%Fall 2010 - Spring 2016.  Graduate teaching assistant.
%\begin{itemize}
%            \setlength\itemsep{0.00em}
%\item Lectured for:
%    \begin{itemize}
%            \setlength\itemsep{0.00em}
%        \item[] The Spirit and Uses of Mathematics (one semester)
%        \item[] Quantitative Reasoning and Math Skills (one semester)
%        \item[] Calculus for Social Science and Business (one semester)
%        \item[] Calculus I (three semesters)
%        \item[] Calculus II (two semesters)
%        \item[] Calculus III (one semester)
%    \end{itemize}
%\item TA/grader for:
%    \begin{itemize}
%            \setlength\itemsep{0.00em}
%        \item[] College Algebra (one semester)
%        \item[] Finite Math for Social Science and Business (one semester)
%        \item[] Calculus I (two semesters)
%        \item[] Calculus III (one semester)
%    \end{itemize}
%\end{itemize}



%\newpage

%\section{\sc Scientific Research Experience}
%
%\begin{tabular}{@{}p{0.7in}p{6in}}
%
%2013--2016 & Applications of Cryptographic Hash Functions \\
%& \hspace{0.2in} Advisor: J. Black, Department of CS (Courtesy Appointment in Math).\\
%& \hspace{0.2in} University of Colorado, Boulder.\\
%\\
%
%2009 & Graph Reconstruction. \\
%& \hspace{0.2in} Advisor: S. G. Hartke, Department of Mathematics. \\
%& \hspace{0.2in} University of Nebraska, Lincoln, Summer Research Program.\\
%\\
%
%2008 & Expected Length of Minimal Spanning Trees.\\
%& \hspace{0.2in} Advisors: P. Otto and C. Starr, Department of Mathematics.\\
%& \hspace{0.2in} Willamette Valley REU-RET Consortium for Mathematics Research.\\
%
%\end{tabular}


\section{\sc Publications}

J. Nishikawa, P. Otto, C. Starr,  \textit{Polynomial Representation for the Expected Length of Minimal Spanning Trees,}
Pi Mu Epsilon Journal, 13 (2012), no. 6, pp 357-365.

S. G. Hartke, H. Kolb, J. Nishikawa, D. Stolee,  \textit{Automorphism Groups of a Graph and a Vertex-Deleted Subgraph,} Electron. J. Combin., 17 (2010), no. 1, R134, 8pp.

J. Nishikawa, \textit{Solution to Problem 1187,} Pi Mu Epsilon Journal, 12 (2009), no. 10, pp 565-566.

%
%\section{\sc Conference \\ Talks}
%
%\emph{A simple piston problem}, $95^ {th} $ Statistical Mechanics
%Conference, Rutgers University. (May 1996)
%
%\emph{A simple piston problem}, Workshop on Dynamical Systems and
%Related Topics, University of Maryland, College Park. (March 1996)







\section{\sc Talks}

\emph{Next Generation Process Emulation with BINEE}, Presentation, CAMLIS - Washington DC. (October 2019)

\emph{Cryptocurrencies, Security, and You}, Full day session, RMISC - Denver, CO. (May 2018)

\emph{A Look Back at OpenSSL and Heartbleed}, Technical training, SnowFROC - Denver, CO. (March 2018)

\emph{All About Bitcoin}, Security 101, SecureSet Academy - Denver, CO.  (April 2017)

\emph{Applications of Cryptographic Hash Functions}, Dissertation defense, University of Colorado - Boulder, CO. (April 2016)

\emph{Bitcoin, Distributed Consensus, and Proof-of-Work}, Hackers Club, University of Colorado - Boulder, CO. (February 2015)

\emph{Hash Functions, a Soft Introduction}, Jim Albaugh Mathematics Colloquium, Willamette University - Salem, OR. (November 2014)

\emph{A Fourier-Analytic Proof of Quadratic Reciprocity}, Slow Pitch Colloquium Series, University of Colorado - Boulder, CO. (May 2012)




%
%\section{\sc Honors and Awards}
%
%\begin{tabular}{@{}p{0.8in}p{4in}}
%2013--2016 & Colorado Graduate Grant \\
%           & University of Colorado, Boulder \\
%2010 & Chester F. Luther Mathematics Award \\
%     & Willamette University \\
%2010 & Phi Beta Kappa Honor Society \\
%2007--2010 & VP of Community Outreach \\
%           & National Society of Collegiate Scholars \\
%2009 & Barry M. Goldwater Scholarship Honorable Mention \\
%
%\end{tabular}
%


%\section{\sc Extended Professional Travel}
%
%\begin{tabular}{@{}p{0.4in}p{0.3in}p{4in}}
%Fall & 1995 & �cole normale sup�rieure de Lyon, Unit� de
%math�matiques pures et appliqu�es, France\\
%
%
%Summer & 1995 & Time at work trimester on
%dynamical systems, Institut Henri Poincar�, Paris, France\\
%\end{tabular}

%\section{\sc Graduate Coursework}
%
%\begin{tabular}{@{}p{2.3in}p{3in}}
%\begin{list1}
%\item Real Analysis
%\item Complex Analysis
%\item Abstract Algebra
%\item Group Theory
%\item Topology
%\item Differential Geometry
%
%\end{list1}
%&
%\begin{list1}
%\item Functional Analysis
%\item Intro Number Theory
%\item Analytic Number Theory
%\item Algebraic Number Theory
%\item Cryptography
%\item Advanced and Random Algorithms
%\end{list1}
%
%\end{tabular}
\section{\sc Relevant \\ Skills}

\begin{tabular}{@{}p{0.8in}p{6in}}

Programming:& Python, Go, Rust, C/C++, Haskell\\
Other:& Bash, Vim, AWS, Docker, GDB, Radare2, SQL, Kubernetes, \LaTeX

\end{tabular}

%
%\section{\sc Professional \\ Associations}
%
%American Mathematical Society (AMS)\\
%Mathematics Association of America (MAA) \\
%Pi Mu Epsilon ($\Pi$ME)


%\section{\sc References}
%
%{\bf Lai-Sang Young}, The Henry and Lucy Moses Professor of Science,
%Courant Institute of Mathematical Sciences, New York University,
%(212)998-3286, \texttt{lsy@cims.nyu.edu}

\end{resume}
\end{document}
