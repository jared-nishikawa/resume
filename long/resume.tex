%% resume from absolute scratch
\documentclass{letter}

% set custom margins
\usepackage[margin=0.5in]{geometry}

% some nice side effects, but
% mostly just to get rid of indentation
\usepackage{parskip}

% set default font family to be sans serif
%\renewcommand{\familydefault}{\sfdefault}

\begin{document}

\newcommand{\vspc}{\vskip5mm}

% remove page number from bottom of page
\pagenumbering{gobble}

% smaller bullets
\renewcommand\labelitemi{\raisebox{0.25ex}{\tiny$\bullet$}}

% name is bold for quick glance
\textbf{\Huge{Jared Nishikawa, PhD}}

\texttt{jared.nishikawa@gmail.com} \\


% headers actually don't need to be bold
% if people look at company names first
\large{Experience}
\hrule

% company names are bold for quick glance
\textbf{Tripleko LLC \hfill 10/18 - \textit{Present}}

Owner - software consultant

\begin{itemize}
    \setlength\itemsep{-0.5em}
    \item Owner, software consultant
\end{itemize}

\textbf{VMware \hfill 10/22 - 11/23}

Senior Information Security Engineer

\begin{itemize}
    \setlength\itemsep{-0.5em}
    \item Vulnerability management
    \item Automation pipelines, tooling, cloud infrastructure
    \item Red team automation
\end{itemize}

\textbf{VMware Carbon Black \hfill 7/18 - 10/22}

Senior Threat Researcher

\begin{itemize}
    \setlength\itemsep{-0.5em}

    \item Wrote a novel malware classifier.  This project involved breaking down binary executables into control flow graphs based on opcodes (following JMP, CALL, RET, for example).  The analyzer then generated an n-gram frequency table, which was used as a manageably short vector to compare and cluster malware families.  Using a pre-existing labeled set of malware, the classifier could then accurately classify new malware samples into one of the existing clusters.

    \item Wrote an automated UPX unpacker.  Malware often comes packed, and the packed binaries commonly are stripped of UPX metadata so while they would still run perfectly fine, the UPX unpacker would not simply unpack them.  This project required an in-depth analysis of the UPX packing algorithm to automatically detect the relevant header information from a stripped, packed file, and then automatically unpack it.  This was used in conjunction with the malware classifier: unpacked files would be more accurately placed into a malware family.

    \item Contributed to an in-house powershell deobfuscator.  This approach was based on a powershell language scanner, parser, and reducer.  This was later incorporated into the Carbon Black product.

    \item Contributed to an in-house (later open-sourced) PE file emulator called Binee (Binary Emulation Environment). The approach extended current emulators by mocking out large parts of Windows OS including: threading, scheduler, registry, file system and various userland process structures.

    \item Technical lead for junior threat analysts. Developed software to automate repeated analyst tasks, including pulling threat metadata, recording which threats have been triaged and by which analyst, and migration away from local CSV files to a central database. Automated detection and dismissal of common false positives.

\end{itemize}

\textbf{SecureSet Academy \hfill 4/16 - 7/18}

Cybersecurity Instructor

\begin{itemize}
    \setlength\itemsep{-0.5em}
    \item Python programming instructor for beginner to intermediate students. Covered basics of functions, classes, file IO, encodings, and socket programming.
    \item Cryptography instructor.  Topics covered: history and technical details of symmetric encryption and validation, stream ciphers, block ciphers (ECB, CBC, CTR), hash functions, MACs, RC4, WEP, DES, 3DES, AES, MD5, SHA1, SHA2, SHA3.  History of major cryptographic attacks, rainbow tables, meet-in-the-middle attacks, padding oracle attacks.  History and technical details of public key encryption and validation, RSA, Diffie-Hellman, SSH, SSL, certificates and certificate authorities, as well as major cryptographic attacks like heartbleed and sslstrip.
    \item Administration of Canvas LMS, including automation of course creation, account creation, student enrollment, instructor access, and grading tools.
\end{itemize}

\vspc

% projects
% binee (binary emulation)
% depack (automated unpacking)
% reveal (powershell deobfuscation: language parsing, abstract syntax trees)
% vertigo (malware similarity analysis, control flow graphs)


\large{Talks}
\vskip1mm
\hrule

\textbf{CAMLIS (Washington DC) \hfill 10/19}

\textit{Next Generation Process Emulation with Binee}

\textbf{RMISC (Denver, CO) \hfill 5/18}

\textit{Cryptocurrencies, Security, and You}

\textbf{SnowFROC (Denver, CO) \hfill 3/18}

\textit{A Look Back at OpenSSL and Heartbleed}

\textbf{SecureSet Academy (Denver, CO) \hfill 4/17}

\textit{All About Bitcoin}

\textbf{Hackers Club - University of Colorado (Boulder, CO) \hfill 2/15}

\textit{Bitcoin, Distributed Consensus, and Proof-of-work}

\textbf{Jim Albaugh Mathematics Colloquium - Willamette University (Salem, OR) \hfill 11/14}

\textit{Hash Functions, a Soft Introduction}

\vspc

\large{Education}
\vskip1mm
\hrule

\textbf{PhD, Mathematics - University of Colorado, Boulder \hfill 8/10 - 5/16}
\begin{itemize}
    \setlength\itemsep{-0.5em}
    \item Dissertation: \textit{Applications of Cryptographic Hash Functions}
    \item Advisor: John Black
\end{itemize}

\textbf{BA, Mathematics - Willamette University \hfill 8/06 - 5/10}

\vspc

\large{Tech}
\vskip1mm
\hrule

\textbf{Comfort level := ``go-to for nearly any task''}

bash, python, go, rust, vim, git, aws, docker, \LaTeX

\vspc

\textbf{Comfort level := ``substantial googling necessary''}

haskell, c/c++, gdb, radare2, javascript, php, sql, kubernetes

\vspc

\textbf{Comfort level := ``have used, no desire to revisit''}

java, html/css, x86 assembly


\end{document}
